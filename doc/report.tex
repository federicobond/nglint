\documentclass[]{article}
\usepackage{lmodern}
\usepackage{amssymb,amsmath}
\usepackage{ifxetex,ifluatex}
\usepackage{xltxtra}
\usepackage{fixltx2e} % provides \textsubscript
\ifnum 0\ifxetex 1\fi\ifluatex 1\fi=0 % if pdftex
  \usepackage[T1]{fontenc}
\else % if luatex or xelatex
	% \usepackage{fontspec}
  % \defaultfontfeatures{Ligatures=TeX,Scale=MatchLowercase}
\fi
% use upquote if available, for straight quotes in verbatim environments
\IfFileExists{upquote.sty}{\usepackage{upquote}}{}
% use microtype if available
\IfFileExists{microtype.sty}{%
\usepackage{microtype}
\UseMicrotypeSet[protrusion]{basicmath} % disable protrusion for tt fonts
}{}
\usepackage{hyperref}
\hypersetup{unicode=true,
            pdftitle={nglint: nginx configuration file linter},
            pdfborder={0 0 0},
            breaklinks=true}
\urlstyle{same}  % don't use monospace font for urls
\usepackage{biblatex}

\addbibresource{bibliography.bib}
\usepackage{listings}
\IfFileExists{parskip.sty}{%
\usepackage{parskip}
}{% else
\setlength{\parindent}{0pt}
\setlength{\parskip}{6pt plus 2pt minus 1pt}
}
\setlength{\emergencystretch}{3em}  % prevent overfull lines
\providecommand{\tightlist}{%
  \setlength{\itemsep}{0pt}\setlength{\parskip}{0pt}}
\setcounter{secnumdepth}{0}
% Redefines (sub)paragraphs to behave more like sections
\ifx\paragraph\undefined\else
\let\oldparagraph\paragraph
\renewcommand{\paragraph}[1]{\oldparagraph{#1}\mbox{}}
\fi
\ifx\subparagraph\undefined\else
\let\oldsubparagraph\subparagraph
\renewcommand{\subparagraph}[1]{\oldsubparagraph{#1}\mbox{}}
\fi
\usepackage{verbatim}
\usepackage{tikz}
\usetikzlibrary{arrows}

\setmonofont[Scale=MatchLowercase]{Monaco}

\lstset{
	aboveskip={1.5\baselineskip},
	backgroundcolor=\color{white},
	basicstyle=\ttfamily,
	breaklines=true,
	captionpos=b,
	columns=fixed,
	commentstyle=\color[rgb]{0,0.2,0},
	extendedchars=true,
	frame=none,
	keywordstyle=\color[rgb]{0,0,1},
	language=Haskell,
	numbers=left,
	numbersep=5pt,
	numberstyle=\scriptsize\ttfamily\color[rgb]{0.5,0.5,0.5},
	prebreak = \raisebox{0ex}[0ex][0ex]{\ensuremath{\hookleftarrow}},
	showspaces=false,
	showstringspaces=false,
	showtabs=false,
	stringstyle=\color[rgb]{0,0.6,0},
	tabsize=4,    
	upquote=true,
	title=\lstname,
}


\title{nglint: nginx configuration file linter}
\providecommand{\subtitle}[1]{}
\subtitle{Final Course Assignment Report}
\author{Federico Bond \\
\texttt{\href{mailto:fbond@itba.edu.ar}{fbond@itba.edu.ar}} }
\date{February 15, 2016}

\begin{document}
\maketitle

\section{Introduction}\label{introduction}

The following report describes the design of a linter for \textbf{nginx}
configuration files, written in Haskell as final course project for the
Functional Programming course at ITBA.

Linters are widely used in the software industry. These tools run static
analysis on code to pinpoint errors and potential trouble spots. They
can be executed manually or configured to run automatically inside a
Continuous Integration server, to ensure that no developer introduces new
issues.

Several linters have been written in Haskell before. The most famous
outside the Haskell ecosystem is
\href{http://www.shellcheck.net/}{Shellcheck}, which finds
problematic constructs in shell scripts. It is worth mentioning that these
tools can also serve an educational purpose, providing contextual advice
to the user in order to improve his or her code. Another known linter is
\href{https://github.com/ndmitchell/hlint}{HLint}, which was used in
this project and provided useful suggestions for simplifying some of the code.

\section{The \textbf{nginx} reverse proxy}\label{nginx}

\textbf{nginx} is a widely used HTTP and reverse proxy server. It can
serve static files over the web and/or proxy web applications, among
other things. It is considered one of the fastest HTTP servers around, and
companies as big as Automattic (which owns \href{https://wordpress.com/}{WordPress.com})
use \textbf{nginx} to power their platforms.

Configuring an \textbf{nginx} server can seem daunting at first. There
are hundreds of options and some of them have grave security
consequences. Novice users could benefit from a tool that made helpful
contextual suggestions in order to harden an improve their server configuration.

\subsection{Configuration syntax definition}\label{syntax-definition}

The \textbf{nginx} configuration uses a C-like syntax consisting of
directives, which let you specify the behaviour of the server, and
blocks, which define new contexts where these directives apply. An
example configuration file is shown below:

\begin{lstlisting}[language=]
user www-data;
pid /var/run/nginx.pid;

events {
  worker_connections 2048;
  use epoll;
}

http {
  # default http server
  server {
    root /srv/website;
  }
}
\end{lstlisting}

In this example, there are two blocks, \texttt{events} and
\texttt{http}, and several directives such as \texttt{user} and
\texttt{worker\_connections}.

A simplified version of this grammar in extended Backus-Naur form is
provided below:

\begin{verbatim}
<config> ::= <declaration>*
<declaration> ::= <directive> | <block>
<directive> ::= <identifier> <arg>* ';'
<block> ::= <identifier> <arg>* '{' <declaration>* '}'
<identifier> ::= <alpha> (<alpha-num> | '_')*
\end{verbatim}

\section{Preliminary Research}\label{preliminary-research}

My first task was researching the topic of punctional parsing.
Many tools exist to build parsers. Lex and Yacc are two very popular
ones that have been around for a very long time. Parsing is usually
divided into two steps. First, a lexer turns a byte stream from a source fileinto a
stream of tokens, and then a proper parser turns this stream of tokens
into an Abstract Syntax Tree (AST) which is a representation of the
structure of the source file inside the program.

Traditional tools for building parsers use their own syntax to describe
the language grammar and the ways that constructs can be combined together
is predetermined by the tool. One alternative is to use Functional parser
combinators, a topic that has gathered the attention of researchers for
many years. These combinators are first-class values within the host language,
which opens the posibility to much richer composition.

In 2001, Daan Leijen and Erik Meijer introduced Parsec, a parser combinator
library for Haskell that was robust enough to be used for real-world projects,
unlike many of the previous implementations, which suffered from space leaks
or inability to report precise error messages\footnote{\url{http://research.microsoft.com/en-us/um/people/daan/download/papers/parsec-paper.pdf}}.
Parsec quickly gained adoption and is now used in more than 700 projects
published in Hackage, the most used Haskell package archive.

The parser combinators from \texttt{parsec} work inside the \texttt{Parser} monad.
A typical parser looks like this:

\begin{lstlisting}
identifier :: Parser String
identifier = do
  c <- identStart
  cs <- many identLetter
  return (c:cs)
\end{lstlisting}

This approach makes it really easy to build more complex parsers by combining
simpler ones. For more realistic examples, take a look at the code Appendix.

\section{Program Architecture}\label{program-architecture}

The linting process can be divided into three steps:

\begin{verbatim}
        Parse -> Lint -> Output
\end{verbatim}

The parsing step takes the file contents as input and produces an
Abstract Syntax Tree (AST) as output.

Once the parser has finished its job, the linter will apply each one of
the rules to the AST, returning a list of hints. Each rule needs to
match certain declarations against the syntax tree. To do this easily, a
set of matchers was built that can be composed together via
\texttt{\textgreater{}\textgreater{}\textgreater{}} (recall that
\texttt{f\ \textgreater{}\textgreater{}\textgreater{}\ g\ =\ g\ .\ f}).
The reader may be curious as to why it is necessary to compose these matchers
in reverse. The answer is that we want to apply the functions in the order in
which they are written. Let's take a look at a simple example:

\begin{minipage}{\linewidth}
\begin{lstlisting}[basicstyle=\ttfamily]
matchRootInsideLocation :: Matcher
matchRootInsideLocation =
    matchBlock "location" >>>
    matchDirective "root"
\end{lstlisting}
\end{minipage}

We want to find the blocks named \texttt{location}, which contain a directive
named \texttt{root} inside. The first function will take a list of directives
and return only the location blocks. The second must then filter and find
the directives named \texttt{root}. This style of composition reads more
natural than:

\begin{lstlisting}[basicstyle=\ttfamily]
matchRootInsideLocation :: Matcher
matchRootInsideLocation =
  matchDirective "root" . matchBlock "location"
\end{lstlisting}

Finally, we pass the list of hints to a formatter, which in the case of
\textbf{nglint} can be configured via command-line options. The
\texttt{ansi-terminal}\footnote{\url{https://hackage.haskell.org/package/ansi-terminal-0.6.2.3}}
library was used for colored console output.

\section{Difficulties encountered}\label{difficulties-encountered}

The most difficult part of this project was to find a proper abstraction
for constructing linter rules, which avoided unnecessary boilerplate. I
considered the use of Monads, Zippers and the \texttt{Traversable} type
class, but finally settled for simple composition of filter functions,
which proved good enough for the task at hand and kept mental overhead
at a reasonable level.

The error output from \texttt{ghc} also proved hard to understand at
first, especially for someone used to traditional stack traces and
syntax errors in imperative / object-oriented code. With practice it
became quite natural, though.

\section{Future extensions}\label{future-extensions}

Throughout the duration of the project, I kept a To Do list with ideas
for future extensions. A sample of them has been reproduced below:

\begin{itemize}
\tightlist
\item
  Add rules which depend on the presence or absence of siblings or
  otherwise related declarations.
\item
  Add command line options for ignoring certain rules.
\item
  Ignore linter errors when preceded by a specially formatted comment.
\item
  Use the QuickCheck library for testing the codebase.
\end{itemize}

\section{Further research}\label{further-research}

I would like to explore more idiomatic alternatives for constructing
linter rules. The \texttt{hlint} and \texttt{shellcheck} projects might
be useful for this task. I would also like to deepen my understanding of
Monad Transformers, which appear useful for reducing the nesting of code
and improving error handling.

\section{Conclusions}\label{conclusions}

I had never written a complete program in Haskell before. The experience
proved highly valuable and encouraged me to use Haskell and other pure
functional programming languages like Elm and PureScript in future
scenarios. I was surprised at how easily it was to reason about the code
at hand.

Having said that, the state of documentation in the Haskell ecosystem
leaves much to be desired. Having to read academic papers to understand
how to use each library is a bit tedious, but you grow used to it over
time. Many libraries could benefit from a concise README with sample
code and common caveats, specially considering that the heavy use of
types in Haskell makes it easy to find what you need from there.

\section{Bibliography}\label{bibliography}

Daan Leijen, Erik Meijer. 2001. “Parsec: Direct Style Monadic Parser Combinators
for the Real World.” \url{http://research.microsoft.com/en-us/um/people/daan/download/papers/parsec-paper.pdf}.

\begin{center}\rule{0.5\linewidth}{\linethickness}\end{center}

\clearpage

\section{Appendix A: Code}\label{appendix-a-code}

\lstinputlisting{../src/Main.hs}
\lstinputlisting{../src/NgLint/Parser.hs}
\lstinputlisting{../src/NgLint/Linter.hs}
\lstinputlisting{../src/NgLint/Messages.hs}
\lstinputlisting{../src/NgLint/Matchers.hs}
\lstinputlisting{../src/NgLint/Position.hs}
\lstinputlisting{../src/NgLint/Rules.hs}
\lstinputlisting{../src/NgLint/Output/Common.hs}
\lstinputlisting{../src/NgLint/Output/Gcc.hs}
\lstinputlisting{../src/NgLint/Output/Pretty.hs}

\end{document}
